Integer factorization serves an important role in many cryptography systems. 
It is about finding two positive integers $p$ and $q$ 
for a given positive composite number $n$
such that $n=pq$ and $1<p\le q<n$.
However, it is a well-known NP-intermediate candidate. 
We still don't have any algorithm to solve it in polynomial time.

Taylor, a number theorist, created another factorization problem as follows.

Given a prime number $p$ and two integers $a_0, a_1\in\{0,1,\dots p-1\}$. 
Find two integers $b_0, b_1\in\{0,1,\dots p-1\}$ such that 
$a_0\equiv b_0\cdot b_1~(\bmod~p)$ and $a_1\equiv b_0+b_1~(\bmod~p)$.

``This factoring is way much cooler, in the sense that it can
be computed efficiently,'' said Taylor. 
Now, he invites you to enjoy this new variant of factorization.
