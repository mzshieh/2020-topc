There are $n$ children (numbered from $1$ to $n$) who are learning
the arithmetic operations, which include \emph{addition} ``$+$'', 
\emph{subtraction} ``$-$'', 
\emph{multiplication} ``$\times$'', and \emph{division} ``$\div$'', on 
rational numbers.
Each child has a paper sheet with only one zero on it. 
Their teacher, Frank, will give out $q$ operations. 
The $i$-th operation consists of an operator $c_i$ and an integer $x_i$.
However, Frank only wants some children to perform the operation.
Only children $\ell_i,\ell_{i+1},\dots,r_i$ are asked to append the 
operator $c_i$ and the number $x_i$ to their paper sheet.
After Frank's assignment, every child has an expression to evaluate.

For example, let $n=3$, $q=2$, $c_1$ be ``$+$'', $x_1=1$, $\ell_1=1$, $r_1=2$, 
$c_2$ be ``$-$'',  $x_2=2$, $\ell_2=2$, $r_2=3$. The expressions of children
$1$, $2$ and $3$ are $0+1$, $0+1-2$ and $0-2$, respectively.

However, Frank is really lazy and wants to verify the answers quickly. 
So he asks you to calculate the sums of the values of all children's 
expressions.
If the value of the expression assigned to child $i$ is $\frac{a_i}{b_i}$, 
then you have to use $a\times b^{-1}\mod 10^9+7$ instead. $b^{-1}$ is any number satisfying 
$b\times b^{-1}\equiv 1\mod 10^9+7$. 
If the sum is greater than $10^9+6$, then return the sum modulo $10^9+7$ to
Frank.

Note: The arithmetic operations has PEMDAS rule, that is, 
Multiplication/Division before Addition/Subtraction.
